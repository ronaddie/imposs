%ieee_jrnl (title), ieee_jrnl_template.tex
%% bare_jrnl.tex
%% V1.2
%% 2002/11/18
%% by Michael Shell
%% mshell@ece.gatech.edu
%%
%% NOTE: This text file uses UNIX line feed conventions. When (human)
%% reading this file on other platforms, you may have to use a text
%% editor that can handle lines terminated by the UNIX line feed
%% character (0x0A).
%%
%% This is a skeleton file demonstrating the use of IEEEtran.cls
%% (requires IEEEtran.cls version 1.6b or later) with an IEEE journal paper.
%%
%% Support sites:
%% http://www.ieee.org
%% and/or
%% http://www.ctan.org/tex-archive/macros/latex/contrib/supported/IEEEtran/
%%
%% This code is offered as-is - no warranty - user assumes all risk.
%% Free to use, distribute and modify.

% *** Authors should verify (and, if needed, correct) their LaTeX system  ***
% *** with the testflow diagnostic prior to trusting their LaTeX platform ***
% *** with production work. IEEE's font choices can trigger bugs that do  ***
% *** not appear when using other class files.                            ***
% Testflow can be obtained at:
% http://www.ctan.org/tex-archive/macros/latex/contrib/supported/IEEEtran/testflow


% Note that the a4paper option is mainly intended so that authors in
% countries using A4 can easily print to A4 and see how their papers will
% look in print. Authors are encouraged to use U.S. letter paper when 
% submitting to IEEE. Use the testflow package mentioned above to verify
% correct handling of both paper sizes by the author's LaTeX system.
%
% Also note that the "draftcls" or "draftclsnofoot", not "draft", option
% should be used if it is desired that the figures are to be displayed in
% draft mode.
%
% This example can be formatted using the peerreview
% (instead of journal) mode.
\documentclass[journal]{IEEEtran}
\hyphenation{op-tical net-works semi-conduc-tor}


\begin{document}
\title{Certain cybersecurity: the impossible dream}

\author{Kaled~Aljebur,
  Mostfa Albdair,
  Ron~Addie,~\IEEEmembership{Member,~IEEE,}
}% <-this % stops a space

\markboth{Journal of \LaTeX\ Class Files,~Vol.~1, No.~11,~November~2002}{Shell \MakeLowercase{\textit{et al.}}: Bare Demo of IEEEtran.cls for Journals}

\maketitle


\begin{abstract}
  The abstract goes here.
\end{abstract}

\IEEEpeerreviewmaketitle



\section{Introduction}
One of the most important themes in the history of the philosophy of science,
initiated by David Hume, has been the difficulty of inductive reasoning,
i.e., how are we able to infer facts from observations.
Karl Popper's ``solution'' to this problem
is that the scientific method is really about trying to find evidence
  {\em against} a hypothesis, and when such evidence cannot be found, despite our best
efforts, this can be regarded as strong evidence {\em for} the hypothesis.
Popper's resolution was, and remains, quite influential.

However, this debate has not ceased since Popper's contribution.
Other key contributions were made by Kuhn, and xx,
but there is still no universally accepted resolution.

However, more pragmatically, this question is also addressed by
statistics. Furthermore, statistics has quantitative procedures.

Even so, statistics also has some really difficult foundational issues
that are not just theoretical. Fundamentally, statistical reasoning relies
on assumptions, just like deductive reasoning. In particular, we have
to adopt a model. Naive practitioners (and that is the vast majority,
esp given that most of them are not actually educated in statistics),
like to believe that it is satisfactory to pick a standard statistical
method and apply it. This is, after all, what they are taught to do. In
fact, not behaving this way is regarded by most users of statistics as
unorthodox, and unsound.

So, statistics is really the modern form of inductive reasoning, and
it is certainly a lot better defined than the philosophical concept of
inductive reasoning.

But, what about deductive reasoning?

The "default" viewpoint is that we only use this in mathematics,
or perhaps for going from one set of assumptions to another. But
cybersecurity, and especially public key cryptography, and also
block-chain techniques, seem to suggest that we can effectively use
deductive reasoning about the real world.

In cybersecurity, the essential problem is to prove that something is
impossible. (You could call it "the impossible dream", i.e. not dreaming
about achieving something impossible, but rather dreaming that we can make
something not wanted impossible). This is, in general, difficult. However,
here is a simple example of how deductive reasoning can easily achieve it.

Suppose I ask you to draw a right-angle triangle, on a flat surface, with
sides of length 3, 4, and 6. (Yes: 6). The right answer to such a request
is to say: "No, I can't do that. It is impossible.". You can show this
by deductive reasoning. But the statement is about a real-world event:
drawing a triangle. This has nothin to do with the precision of the
measurements. The inaccuracy can be quantified. Even an approximately
3,4,6 right-triangle is impossible. There are events, even with
approximate measurements, which can be proved to be impossible. Granted,
the impossibility of non-pythagorean triangles doesn't seem to help a
lot in cybersecurity, but what if we can entangle real world events in
mathematically precise statements in such a way that the events themselves
become impossible?

We can, and already do this, using similar reasoning to the above,
concluding that certain cybersecurity events can't happen: this
is what is happening when certificates are used, and digital
signatures. Conventional wisdom, from the background of science,
statistics, and experimental methods, strongly suggests that certainty
of this sort is impossible. But the triangle shows that the problem is
not with deducing impossibility. Deducing impossibility is possible! But
can we employ such deductions?

I think we are already doing so, and there will be a lot more such deductions in the near future.


This demo file is intended to serve as a ``starter file"
for IEEE journal papers produced under \LaTeX\ using IEEEtran.cls version
1.6b and later.

May all your publication endeavors be successful.

\hfill mds

\hfill November 18, 2002

\section{Problem Statement}
Subsection text here.



\section{Literature Review}
Subsubsection text here.

\section{Analysis}
Subsubsection text here.


\section{Conclusion}
The conclusion goes here.

\appendices
\section{Proof of the First Zonklar Equation}
Appendix one text goes here.

\section{}
Appendix two text goes here.



\begin{thebibliography}{1}

  \bibitem{IEEEhowto:kopka}
  H.~Kopka and P.~W. Daly, \emph{A Guide to {\LaTeX}}, 3rd~ed.\hskip 1em plus
  0.5em minus 0.4em\relax Harlow, England: Addison-Wesley, 1999.

\end{thebibliography}

\end{document}


