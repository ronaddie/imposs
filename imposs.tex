%ieee_jrnl (title), ieee_jrnl_template.tex
%% bare_jrnl.tex
%% V1.2
%% 2002/11/18
%% by Michael Shell
%% mshell@ece.gatech.edu
%%
%% NOTE: This text file uses UNIX line feed conventions. When (human)
%% reading this file on other platforms, you may have to use a text
%% editor that can handle lines terminated by the UNIX line feed
%% character (0x0A).
%%
%% This is a skeleton file demonstrating the use of IEEEtran.cls
%% (requires IEEEtran.cls version 1.6b or later) with an IEEE journal paper.
%%
%% Support sites:
%% http://www.ieee.org
%% and/or
%% http://www.ctan.org/tex-archive/macros/latex/contrib/supported/IEEEtran/
%%
%% This code is offered as-is - no warranty - user assumes all risk.
%% Free to use, distribute and modify.

\documentclass[journal]{IEEEtran}
\hyphenation{op-tical net-works semi-conduc-tor}
\usepackage{enumerate}
\usepackage{amsmath}
\usepackage[dvipsnames]{xcolor}

%Colours code
\def\va#1{\hbox{\color{red}\it #1}}
\def\pr#1{\hbox{\color{ForestGreen}\rm #1}}
\def\co#1{\hbox{\color{blue}\it #1}}
\def\mo#1{\hbox{\color{Purple}\bf #1}}
\def\v{\color{red}}
\def\p{\color{ForestGreen}}

\begin{document}
\title{Certain cybersecurity: the impossible dream}

\author{Kaled~Aljebur,
  Mostfa Albdair,
  Ron~Addie,~\IEEEmembership{Member,~IEEE,}
}% <-this % stops a space

\markboth{}{}

\maketitle


\begin{abstract}
  The abstract goes here.
\end{abstract}

\IEEEpeerreviewmaketitle



\section{Introduction}
One of the most important themes in the history of the philosophy of science,
initiated by David Hume, has been the difficulty of inductive reasoning,
i.e., how are we able to infer facts from observations.
Karl Popper's ``solution'' to this problem
is that the scientific method is really about trying to find evidence
  {\em against} a hypothesis, and when such evidence cannot be found, despite our best
efforts, this can be regarded as strong evidence {\em for} the hypothesis.
Popper's resolution was, and remains, quite influential.

However, this debate has not ceased since Popper's contribution.
Other key contributions were made by Kuhn, and xx,
but there is still no universally accepted resolution.

However, more pragmatically, this question is also addressed by
statistics. Furthermore, statistics has quantitative procedures.

Even so, statistics also has some really difficult foundational issues
that are not just theoretical. Fundamentally, statistical reasoning relies
on assumptions, just like deductive reasoning. In particular, we have
to adopt a model. Naive practitioners (and that is the vast majority,
esp given that most of them are not actually educated in statistics),
like to believe that it is satisfactory to pick a standard statistical
method and apply it. This is, after all, what they are taught to do. In
fact, not behaving this way is regarded by most users of statistics as
unorthodox, and unsound.

So, statistics is really the modern form of inductive reasoning, and
it is certainly a lot better defined than the philosophical concept of
inductive reasoning.

But, what about deductive reasoning?

The "default" viewpoint is that we only use this in mathematics,
or perhaps for going from one set of assumptions to another. But
cybersecurity, and especially public key cryptography, and also
block-chain techniques, seem to suggest that we can effectively use
deductive reasoning about the real world.

In cybersecurity, the essential problem is to prove that something is
impossible. (You could call it "the impossible dream", i.e. not dreaming
about achieving something impossible, but rather dreaming that we can make
something not wanted impossible). This is, in general, difficult. However,
here is a simple example of how deductive reasoning can easily achieve it.

Suppose I ask you to draw a right-angle triangle, on a flat surface, with
sides of length 3, 4, and 6. (Yes: 6). The right answer to such a request
is to say: "No, I can't do that. It is impossible.". You can show this
by deductive reasoning. But the statement is about a real-world event:
drawing a triangle. This has nothing to do with the precision of the
measurements. The inaccuracy can be quantified. Even an approximately
3,4,6 right-triangle is impossible. There are events, even with
approximate measurements, which can be proved to be impossible. Granted,
the impossibility of non-pythagorean triangles doesn't seem to help a
lot in cybersecurity, but what if we can entangle real world events in
mathematically precise statements in such a way that the events themselves
become impossible?

We can do this, and already do it, using similar reasoning to the above,
and in this way we are able to conclude that certain cybersecurity events can't happen: for
example, this happens when certificates are used, and digital
signatures. Conventional wisdom, from the background of science,
statistics, and experimental methods, strongly suggests that certainty
of this sort is impossible. But the triangle shows that the problem is
not with deducing impossibility. Deducing impossibility is possible! What
is not clear is whether we can systematically employ such deductions for useful
cybersecurity purposes, like preventing servers ferom being hacked.

In fact, we are already doing this, and there will be a lot more such
deductions in the near future.

\section{Problem Statement}

Many problems in cybersecurity can be expressed as the requirement that certain
events, or outcomes, should be {\em impossible}.
For example, it is often an objective of an organisation with clients (or
customers), that client data is not accessible except when used for the purpose
of clients. In brief, invalid access to client data should be {\em impossible}.

The problem we tackle in this paper is:
\begin{quote}\em
  How can we systematically, and rigorously, ensure, that
  certain outcomes are impossible?
\end{quote}


\section{Literature Review}

\subsection{The Importance of Security Policies}
According to \cite{glasgow1992logic} the security reasoning can
be needed according to the needed security environment.  Not all
organisations have developed their own version of security policies
\cite{paananen2020state}. More research still needed to distinguish what
could be a good security policies.The measure of what could be a good
security policies should be state clearly.

\subsection{Policies Development}
Policies development it has to be efficient for the organisation security
requirements. The security requirements should be reflected in the
designed security policies.

\subsection{Security breaches recent examples}
Below some of the examples of the recent security breaches

\subsubsection{Optus Security Breach}
Security policy take pivotal position in providing the needed data security. Optus security breach in
September 2022 was one of the very well known of. Around 10 millions data has been exposed to the attachers.
Maintaining and developing the suitable policy may control the security level again such breach
\cite{biddle2022data} \cite{biddle2022p}. According to \cite{paul2022} \cite{john2022} an authenticated API
was the breach used by the attacker. A testing API was used without authentication opened the way for millions
of records belong to the company customers.

\subsubsection{Medibank Security Breach}
Optus security breach is not the fist and nor the last with this huge security impact national wide in Australia.
Medibank which is the largest health insurer in the country been under heavy attack of data security breach.in October
2022 Medibank security breach resulted exposing 9.7 millions of data to the attacker. The available security policy
didn't help to prevent such attack \cite{biddle2022p}.


\section{Examples}

\subsection{Certificate Validation}

\subsubsection{Objectives}

The objectives of a system for validating certificates are:
\begin{enumerate}[CO-1]
  \item\label{validates1} The system reports as valid, certificates that are valid;
  \item\label{validates2} The system reports as invalid, certificates that are invalid;
\end{enumerate}

\subsubsection{Validation by Testing}

If the software implementing the certificate validation is treated
as a black box, and no constraints whatsoever are placed on how it is
implemented, no amount of testing can effect a proof of any
property of the system. It might even be the case that the system
behaves differently at different times.

This is, in effect, the black swan problem, in the context of cybersecurity.

\subsubsection{Validation by Testing of Logically Constrained Software}

Now suppose the software is not a black box, but instead, there
are certain valid rules that apply to its operation. In this case,
it might be the case that a small amount of additional testing
can ensure that the service meets one or more of the objectives.
Consider, in particular, CO-\ref{validates1}, and let us suppose
that \dots

\subsubsection{Impossibility of Certificate Fraud}
Objectives may not be always able to be tested. The impossible of
creating a fake certificated may cannot be completely tested.
Such objective need to be proven logically and mathematically.
Proving CO-\ref{validates2} is not testable and fully experimental.
Thus logical proof will be required.

\subsubsection{Formal Objectives}
For CO-\ref{validates1} the predicated based formal objective may be:

\begin{flalign}\nonumber
   & \forall \pr{~IsValid}(\va{certificate}) \supset \pr{system} \mo{~Can} \\
   & \pr{Report}(\pr{IsValid}(\va{certificate}))
\end{flalign}

For CO-\ref{validates2} the predicated based formal objective may be:

\begin{flalign}\nonumber
   & \forall \neg \pr{~IsValid}(\va{certificate}) \supset \pr{system} \mo{~Can} \\
   & \pr{Report}(\neg \pr{~IsValid}(\va{certificate}))
\end{flalign}

\subsection{Access to Client Data}

\subsubsection{Objectives}

The objectives of a system for controlling access to client data are:
\begin{enumerate}[DO-1]
  \item The system allows clients to update their own data;
  \item The system prevents anyone except the client themselves from updating their data.
\end{enumerate}

\subsubsection{Validation by Testing}

As in the case of certificate validation,
if the software implementing the certificate validation is treated
as a black box, and no constraints whatsoever are placed on how it is
implemented, no amount of testing can effect a proof of any
property of the system.

\subsubsection{Validation by Testing of Logically Constrained Software}

\subsubsection{Impossibility of Alteration of Client Data by someone other than the client}

\subsection{Ping of Death}
Ping of death (PoD) is one of the well-known attacks against any host. PoD uses a TCP handshake
scenario to insert the malicious request. In PoD, the attacker will continue sending the SYN flag
without sending ACK flack which will make the victim think that there is a transmission error. To
prove protection against such attacks, traffic analysis will be needed. Intrusion Detection System
(IDS) will be an ideal option to provide such security analysis. Still, the assumption that only
traffic security analysis will provide the right judgment may not be enough. Such assumptions need
to be proven logically and mathematically.


\subsection{Turkey Dilemma}
There is a concept that can be assumed here. People normally keep feeding a Turkey to be eaten later at
a festival or party. The viewpoint of Turkey is that those people are very kind and generous, but in fact,
Turkey still doesn't know the waiting destiny. We can borrow this example for what can be considered tested
security solutions. Thus, logical and mathematical prove still required.
\section{Conclusion}



\bibliographystyle{plain}
\bibliography{imposs.bib}

\end{document}


